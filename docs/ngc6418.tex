%%%%%%%%%%%%%%%%%%%%%%%%%%%%%%%%%%%%%%%%%
% NGC6418 Documentation
%
% LaTeX template borrowed under creative commons
%
% Author: Lucas Shadler
%
%
%%%%%%%%%%%%%%%%%%%%%%%%%%%%%%%%%%%%%%%%%

%----------------------------------------------------------------------------------------
%	PACKAGES AND OTHER DOCUMENT CONFIGURATIONS
%----------------------------------------------------------------------------------------

\documentclass[twoside,twocolumn]{article}

\usepackage{blindtext} % Package to generate dummy text throughout this template 

\usepackage[sc]{mathpazo} % Use the Palatino font
\usepackage[T1]{fontenc} % Use 8-bit encoding that has 256 glyphs
\linespread{1.05} % Line spacing - Palatino needs more space between lines
\usepackage{microtype} % Slightly tweak font spacing for aesthetics

\usepackage[english]{babel} % Language hyphenation and typographical rules

\usepackage[hmarginratio=1:1,top=32mm,columnsep=20pt]{geometry} % Document margins
\usepackage[hang, small,labelfont=bf,up,textfont=it,up]{caption} % Custom captions under/above floats in tables or figures
\usepackage{booktabs} % Horizontal rules in tables

\usepackage{lettrine} % The lettrine is the first enlarged letter at the beginning of the text

\usepackage{enumitem} % Customized lists
\setlist[itemize]{noitemsep} % Make itemize lists more compact

\usepackage{abstract} % Allows abstract customization
\renewcommand{\abstractnamefont}{\normalfont\bfseries} % Set the "Abstract" text to bold
\renewcommand{\abstracttextfont}{\normalfont\small\itshape} % Set the abstract itself to small italic text

\usepackage{titlesec} % Allows customization of titles
\renewcommand\thesection{\Roman{section}} % Roman numerals for the sections
\renewcommand\thesubsection{\roman{subsection}} % roman numerals for subsections
\titleformat{\section}[block]{\large\scshape\centering}{\thesection.}{1em}{} % Change the look of the section titles
\titleformat{\subsection}[block]{\large}{\thesubsection.}{1em}{} % Change the look of the section titles

\usepackage{fancyhdr} % Headers and footers
\pagestyle{fancy} % All pages have headers and footers
\fancyhead{} % Blank out the default header
\fancyfoot{} % Blank out the default footer
\fancyhead[C]{Running title $\bullet$ May 2016 $\bullet$ Vol. XXI, No. 1} % Custom header text
\fancyfoot[RO,LE]{\thepage} % Custom footer text

\usepackage{titling} % Customizing the title section

\usepackage{hyperref} % For hyperlinks in the PDF

%----------------------------------------------------------------------------------------
%	TITLE SECTION
%----------------------------------------------------------------------------------------

\setlength{\droptitle}{-4\baselineskip} % Move the title up

\pretitle{\begin{center}\Huge\bfseries} % Article title formatting
\posttitle{\end{center}} % Article title closing formatting
\title{X-Ray Based Analysis of the ``Changing-Look'' Galaxy NGC6418} % Article title
\author{%
\textsc{Lucas Shadler} \\ %\thanks{A thank you or further information} \\[1ex] % Your name
\normalsize Rochester Institute of Technology \\ % Your institution
\normalsize \href{mailto:lxs2208@rit.edu}{lxs2208@rit.edu} % Your email address
\and % Uncomment if 2 authors are required, duplicate these 4 lines if more
\textsc{Dr. Andrew Robinson} \\[1ex] % Second author's name
\normalsize Rochester Institute of Technology \\ % Second author's institution
\normalsize \href{mailto:axrsps@rit.edu}{axrsps@rit.edu} % Second author's email address
\and % Uncomment if 2 authors are required, duplicate these 4 lines if more
\textsc{Dr. Triana Almeyda} \\[1ex] % Second author's name
\normalsize Rochester Institute of Technology \\ % Second author's institution
\normalsize \href{mailto:tra3595@g.rit.edu}{tra3595@g.rit.edu} % Second author's email address
}
\date{\today} % Leave empty to omit a date
\renewcommand{\maketitlehookd}{%
\begin{abstract}
\noindent NGC6418 is a Seyfert I Active Galactic Nucleus (AGN) that
has exhibited qualities of a rare ``changing-look''galaxy. Infrared
and optical observations in the B and V optical bands have 
demonstrated that $3.6$ and $4.5 \mu m$ flux variations lag behind the optical 
continuum with large variations in intrinsic luminosity. This behavior
could be consistent with dust obscuration from a clumpy torus or a 
dusty cloud traversing the line of sight. The nature of this obscurations
can be further quantified through analysis of the X-ray spectrum, 
which will provide insight into the derived spectral shape, X-ray luminosity, and
hydrogen column density.
Recent observational data from the X-ray Multi-mirror Mission (XMM) 
Newton telescope will be reduced and extracted through open-source 
analysis software to develop further conclusions on NGC6418.
\end{abstract}
}

%----------------------------------------------------------------------------------------

\begin{document}

% Print the title
\maketitle

\section{Introduction}

An active galactic nucleus (AGN) is the compact center of a galaxy which typically contributes a significant portion of the galaxy's luminosity. Under the AGN unification scheme, it is believed that AGN are actually supermassive black holes centered in the galaxy, obscured by a dusty toroidal component. Galaxies can be further clasified as Seyfert I, where the spectrum is broadened by obscuration, and Seyfert II, where the spectrum lines are very narrow and prominent. Seyfert classification was further classified by Donald Osterbrock in $1981$, giving rise to Seyfert 1.2,1.5,1.7, and 1.9.

A galaxy's properties can change over time in a way that affects the optical spectrum. A rare group of AGN experience significant changes in their spectrum such that their Seyfert classification changes. These galaxies are named ``changing-look'', and they remain the subject of further study.
\section{NGC6418}

NGC6418 is a Seyfert I galaxy that has exhibited the qualities of a rare ``changing-look'' AGN. Infrared and optical observations in the B and V optical bands from the Spitzer Space Telescope have 
demonstrated that $3.6$ and $4.5 \mu m$ flux variations lag behind the optical 
continuum with large variations in intrinsic luminosity. These observations took data
as NGC6418 shifted from a Seyfert 1.9 to a Type 1 and back within a 3 year span. In a observation of this character the variations are typically attributed to an AGN flare, where the luminosity undergoes a temporary increase or decrease in luminosity, or a change in the obscuration in the line of sight. However, the coherence in the optical and IR variations imply that both flare and obscuration change may be contributing to the observations through the sublimation of dust.

\section{Proposed Work}

Data of the X-ray spectrum after NGC6418 had transitioned back to a Seyfert Type 1.9 has been collected from XMM-Newton. Aside from the XMM-Newton pre-processing dataset, the data needs to be reduced in order to extract significant quantities characterizing the spectral shape and conditions. Several open-source software packages, including XSpec provided by NASA, allow for the reduction of the data set. We will produce analytical quantities such as the Hydrogen column density and the X-ray luminosity so that the final transition state can be compared to the conditions before and during the ``changing-look'' transition. This will provide insight into the mechanism of the ``changing-look'' behavior.


\end{document}
